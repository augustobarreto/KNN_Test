\documentclass{article}
\usepackage[utf8]{inputenc}



\date{\today}

\begin{document}

\textbf{Método KNN}
\vspace{2.0cm}


 O tradicional algoritmo de classificação k-NN, encontra os K vizinhos mais próximos e classifica dados numéricos calculando a distância entre as amostras teste e as amostras de treinamento utilizando a distância euclidiana. [1].
 
 \vspace{1.0cm}

Este método (kNN) é um simples, porém, efetivo para classificação supervisionada. Seus maiores gargalos  dizem respeito a (1) sua baixa eficiência – sendo um método de aprendizagem preguiçoso, limitando a sua aplicação em casos de mineração web dinâmica em grandes repositórios, e (2) sua dependência na seleção de um "bom valor" para k [2]

\vspace{1.0cm}

No trabalho, foi possível observar a baixa eficiência do método utilizando a distância euclidiana, precisando de um grande conjunto de amostras de treinamento para melhorar a sua perfomance.

\vspace{1.0cm}

Referências

\vspace{1.0cm}

[1] Tan P-N (2018) Introduction to data mining. Pearson Education, Chennai

\vspace{1.0cm}
[2] KNN Model-Based Approach in Classification - Gongde GuoHui WangDavid BellYaxin BiKieran Greer



\end{document}